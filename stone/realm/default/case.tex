[interleaved-nps]
太郎  と 花子  だけ が  駅  から 歩い  た。
たろう - はなこ -  NOM えき ABL ある  PST
Only Taro and Hanako walked from the train station.

[hallmark:multi-noms]
太郎  が   母親    が   評判      が 良い。
たろう NOM ははおや NOM ひょうばん NOM い
Taro is such that his mother has a good reputation.

[nom:ga]
雨  が   降り 出し た。
あめ NOM ふ   で PST
It started to rain

風  が   涼 しい。
かぜ NOM すず
The wind is cool.

誰  が   幹事   です か。
だれ NOM かんじ  COP SFP
Who is the secretary?

魚    は、 猫   が  食べ た。
さかな TOP ねこ NOM た  PST
The cat ate the fish.

[nom:ga:nsub]
私    は   数学    が   大好き だ。
わたし TOP すうがく NOM だいす COP
I love mathematics.

いま、 冷  たいビール が  飲み たい なあ。
-    つめ -       NOM  の  Wish SFP
At the moment, (I) want to drink a cold beer.

太郎   が この 本   が  読  み 易 い/ 難 い。
たろう NOM -  ほん NOM よう - やす - にく
For Taro to read this book is easy/difficult.

[nom:ga:ambiguity]
象   は、 鼻  が   長 い。
ぞう TOP はな NOM なが
Speaking of the elephant, its nose is long.

[acc:o]
今   の  若者    は  みんな スマホ を  使い ます。
いま GEN わかもの TOP -    -     ACC つか POL
All young people today use smartphones.

18 歳  の とき、 故郷    を   / から 離れ ました。
- さい GEN -    こきょう ACC - ABL はな POL.PST
At the age of 18, (I) left my hometown.

鳶   が  紺碧     の  空  を  飛び ました。
とび NOM こんぺき GEN そら ACC と   POL.PST
A black kite flew over the cerulean blue sky.

[acc:o:inanimate]
血 が   傷口   から 出 た。
ち NOM きずぐち ABL で PST
Blood went out of the wound.

[gen:no]
今日   の  テレビ 東京  の 『TXNニュース』 の  番組   は  つまらない。
きょう GEN - とうきょう GEN -        GEN ばんぐみ TOP -
Today's "TXN News" programs of TV Tokyo are boring.

ドイツ の スチール の 包丁   が      欲しい です。
-    GEN -     GEN ほうちょう NOM ほ   COP.POL
(I) want a steel kitchen knife made in Germany.

14000 円  の  日本語   の  本   を  買い ました。
-    えん GEN にほんご GEN ほん ACC か   PST
(I) bought a book of Japanese for ¥14000.

[gen:no:ambiguity]
私    の   彼女    の  学校     の  学生
わたし GEN かのじょ GEN がっこう GEN がくせい
the student in the same school as my girlfriend's
the student in the school owned by my girlfriend

[gen:no:apposition]
こちら は 私     の  彼女    の  高橋   さん です。
-    TOP わたし GEN かのじょ GEN たかはし - COP.POL
This is my girlfriend, Takahashi.

こちら は  私    の  彼女    で、  日本人    で、  芸術家      で、  高橋   さん です。
-    TOP わたし GEN かのじょ CONJ にほんじん CONJ げいじゅつか CONJ たかはし - COP.POL
This is my girlfriend, Takahashi, who is a Japanese artist.

[gen:no:transform]
象 は、 鼻 が  長い。
-  TOP - NOM

象   の   鼻  が   長 い。
ぞう GEN はな NOM なが
The nose of elephant is long.

[gen:no:attributive]
太郎   が いつも の  バス で    帰宅し た。
たろう NOM -   GEN  - INSTR きたく PST
Taro went home by the bus he always rode. [adv. + の]

太郎   が  神社    へ   の  道   を    見つけ た。
たろう NOM じんじゃ ALL GEN みち INSTR み    PST
Taro found a way to the shrine. [postpositional phrase + の]

[gen:no:nominalizer]
太郎   が  独 りで 外出 する   の    を  止め た。
たろう NOM ひと - がいしゅつ - NMLZ ACC や PST
Taro stopped going out alone.

どんな 鞄   が   良いですか。 丈夫    の  を ください。
-    かばん NOM い         じょうぶ One ACC Wish
What kind of bag do you want? (I'd like) the durable one, please.

[dat:ni]
先生    に テスト の 範囲    を  聞き ます。
せんせい DAT -   GEN はんい ACC き COP.POL
I will ask for the test scope from teacher. [source]

交換留学生           の  鄭 さん に 会い ました    か。
こうかんりゅうがくせい GEN てい - DAT あ  POL.PST SFP
Have (you) met Mr.Chung, the exchange student? [addressee]

[dat:ni:give]
七夕、     鵲    が    牽牛   と  織女  に  会う 機会  を   与 えます。
たなばた かささぎ NOM けんぎゅう - おりめ DAT あ  きかい ACC あた POL
At the Star Festival, Magpies give a chance to Altair and Vega to meet. [recipient]

7月7日、       鵲    が   彦星   と  織姫   (  の   会合  )  に   鵲橋       を  作 って あげ ます。
しちがつなのか かささぎ NOM ひこぼし - おりひめ - GEN かいごう - DAT じゃっきょう ACC つく - Give POL
On July 7th, Magpies make a bridge for (the meeting of) Altair and Vega. [benefactive]

[dat:ni:all]
昨日   の  午後、 図書館    に   行き ませんでした。
きのう GEN ごご   としょかん DAT い  POL.NEG.PST
(I) didn't go to the library yesterday afternoon. [destination]

学校    に   授業     を   受け に 行き ます。
がっこう DAT じゅぎょう ACC う  DAT い COP.POL
(I) go to school to take classes. [destination][purpose]

学校    に   授業      に  行き ます。
がっこう DAT じゅぎょう DAT い COP.POL
(I) go to school to teach. [destination][purpose]

[dat:ni:loc]
国   の  周り  に  高い 城壁      を  作っ た。
くに GEN まわ DAT たか じょうへき ACC つく PST
There are high walls made around the country.
(Someone) made high walls around the country. [ambiguity]
(Someone) made high walls and installed them around the country. [verbose]

[dat:ni:exist]
机    の   下  に   犬   が   います。
つくえ GEN した DAT いぬ NOM Exist.POL
There is a dog under the table. [animate]

すみません、お 手洗 い は どこ に  あります    か。
-          てあら - TOP -  DAT Exist.POL SFP
Excuse me, where is the washroom? [ianimate]

[loc:de]
国   の  周り  で  高い 城壁      を  作っ た。
くに GEN まわ LOC たか じょうへき ACC つく PST
(Someone) made high walls around the country (, but no information about walls).

[cp:omittable]
太郎   が  今朝 大学    に   行った?
たろう NOM けさ だいかく DAT い PST
Did Taro go to university this morning?

お握り を   外  で      食べる の?
にぎ   ACC そと INSTR  た SFP
Are we going to eat rice balls outside?
